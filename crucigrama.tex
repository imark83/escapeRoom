\documentclass[a4paper]{article}
\usepackage[spanish]{babel}
\usepackage[utf8x]{inputenc}
\usepackage{fullpage}
\usepackage{tikz}

\pagestyle{empty}

\begin{document}
{\LARGE
\begin{itemize}
	\item $-2a-2=-9-a$
	\item $34-b = 3(6+b)$
	\item $c+8=24-c$
	\item $3(d-1)-2(d-2) = 10$
	\item $1999e + 1 = 2000$
	\item $2f+5=11$
	\item $125g=625$
	\item $2(h-1)=10$
	\item $j^3=8$
\end{itemize}
}
\newpage
\centering\begin{tikzpicture}[scale=0.8, x=1cm,y=-1cm]

  \fill[black!18!white] (2,4) rectangle +(1,1);
  \fill[black!18!white] (4,5) rectangle +(1,1);
  \fill[black!18!white] (5,7) rectangle +(1,1);
  \fill[black!18!white] (7,8) rectangle +(1,1);
  \fill[black!18!white] (9,3) rectangle +(1,1);
  \fill[black!18!white] (11,6) rectangle +(1,1);
  \fill[black!18!white] (12,1) rectangle +(1,1);
  \fill[black!18!white] (15,3) rectangle +(1,1);
  \fill[black!18!white] (17,5) rectangle +(1,1);




	\draw[very thick](0,7) node[below right] {\tiny \textbf{8}} grid + (6,1);
	\draw[very thick](2,4) node[below right] {\tiny \textbf{6}} grid + (11,1);
	\draw[very thick](8,1) node[below right] {\tiny \textbf{3}} grid + (6,1);
	\draw[very thick](8,6) node[below right] {\tiny \textbf{7}} grid + (11,1);
	\draw[very thick](7,8) node[below right] {\tiny \textbf{9}} grid + (6,1);

	\draw[very thick](9,0) node[below right] {\tiny \textbf{1}} grid + (1,12);
	\draw[very thick](17,0) node[below right] {\tiny \textbf{2}} grid + (1,9);
	\draw[very thick](15,2) node[below right] {\tiny \textbf{4}} grid + (1,5);
	\draw[very thick](4,3) node[below right] {\tiny \textbf{5}} grid + (1,6);


\end{tikzpicture}

\vspace{2cm}
\begin{itemize}
	\item[a)] Lugar donde se inventó la escritura por primera vez.
	\item[b)] La mejor profesora de lengua y literatura.
	\item[c)] La mejor profesora de historia.
	\item[d)] Rey del antiguo Egipto.
	\item[e)] Famoso escultor aragonés que da nombre a cierto instituto (nombre y apellido seguidos).
	\item[f)] El mejor profesor de matemáticas.
	\item[g)] Enorme piedra clavada en el suelo, cuyas funciones eran la funeraria y la de marcar límites territoriales.
	\item[h)] Etapa de la historia antes de la invención de la escritura.
	\item[j)] Etapa de la historia en la que se practica por primera vez la agricultura.
\end{itemize}
\end{document}
