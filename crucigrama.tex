\documentclass[a4paper]{article}
\usepackage[spanish]{babel}
\usepackage[utf8x]{inputenc}
\usepackage{fullpage}
\usepackage{tikz}

\pagestyle{empty}

\begin{document}
{\LARGE
\begin{itemize}
	\item $-2a-2=-9-a$
	\item $34-b = 3(6+b)$
	\item $c+8=24-c$
	\item $3(d-1)-2(d-2) = 10$
	\item $1999e + 1 = 2000$
	\item $2f+5=11$
	\item $125g=625$
	\item $2(h-1)=10$
	\item $j^3=8$
	\item $\log_2(1024) = m$
\end{itemize}
}
\newpage
\centering\begin{tikzpicture}[scale=0.8, x=1cm,y=-1cm]

  \fill[black!18!white] (1,7) rectangle +(1,1);
  \fill[black!18!white] (4,8) rectangle +(1,1);
  \fill[black!18!white] (6,4) rectangle +(1,1);
  \fill[black!18!white] (7,7) rectangle +(1,1);
  \fill[black!18!white] (9,4) rectangle +(1,1);
  \fill[black!18!white] (10,7) rectangle +(1,1);
  \fill[black!18!white] (12,3) rectangle +(1,1);
  \fill[black!18!white] (14,2) rectangle +(1,1);
  \fill[black!18!white] (15,4) rectangle +(1,1);
  \fill[black!18!white] (16,6) rectangle +(1,1);
  \fill[black!18!white] (17,8) rectangle +(1,1);




	\draw[very thick](5,4) node[below right] {\tiny \textbf{6}} grid + (13,1);
	\draw[very thick](16,6) node[below right] {\tiny \textbf{8}} grid + (3,1);
	\draw[very thick](0,7) node[below right] {\tiny \textbf{9}} grid + (5,1);
	% \draw[very thick](2,10) node[below right] {\tiny \textbf{10}} grid + (6,1);


	\draw[very thick](4,6) node[below right] {\tiny \textbf{7}} grid + (1,6);
	\draw[very thick](7,2) node[below right] {\tiny \textbf{3}} grid + (1,10);
	\draw[very thick](10,2) node[below right] {\tiny \textbf{4}} grid + (1,11);
	\draw[very thick](12,1) node[below right] {\tiny \textbf{2}} grid + (1,12);
	\draw[very thick](14,0) node[below right] {\tiny \textbf{1}} grid + (1,8);
	\draw[very thick](17,3) node[below right] {\tiny \textbf{5}} grid + (1,6);

\end{tikzpicture}

\vspace{2cm}

\begin{itemize}
	\item[a)] El mejor profesor de matemáticas del universo.
	\item[b)] Etapa de la historia antes de la invención de la escritura.
	\item[c)] Rey del imerio ruso.
	\item[d)] País en el que triunfó la Revolución en 1917.
	\item[e)] País que llegó tarde al reparto colonial.
	\item[f)] Sistema defensivo característico de la Primera Guerra mundial.
	\item[g)] Rey del antiguo Egipto.
	\item[h)] Nombre con el que se conoce a la alianza entre Alemania, Imperio Austro-húngaro e Italia.
	\item[j)] Famoso escultor aragonés que da nombre a cierto instituto (nombre y apellido seguidos).
	% \item[m)] La mejor profesora de geografía e historia.
\end{itemize}
\end{document}
